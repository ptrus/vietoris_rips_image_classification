%%%%%%%%%%%%%%%%%%%%%%%%%%%%%%%%%%%%%%%%%
% Programming/Coding Assignment
% LaTeX Template
%
% This template has been downloaded from:
% http://www.latextemplates.com
%
% Original author:
% Ted Pavlic (http://www.tedpavlic.com)
%
% Note:
% The \lipsum[#] commands throughout this template generate dummy text
% to fill the template out. These commands should all be removed when 
% writing assignment content.
%
% This template uses a Perl script as an example snippet of code, most other
% languages are also usable. Configure them in the "CODE INCLUSION 
% CONFIGURATION" section.
%
%%%%%%%%%%%%%%%%%%%%%%%%%%%%%%%%%%%%%%%%%

%----------------------------------------------------------------------------------------
%	PACKAGES AND OTHER DOCUMENT CONFIGURATIONS
%----------------------------------------------------------------------------------------

\documentclass{article}

\usepackage{fancyhdr} % Required for custom headers
\usepackage{lastpage} % Required to determine the last page for the footer
\usepackage{extramarks} % Required for headers and footers
\usepackage[usenames,dvipsnames]{color} % Required for custom colors
\usepackage{graphicx} % Required to insert images
\usepackage{listings} % Required for insertion of code
\usepackage{courier} % Required for the courier font
\usepackage{lipsum} % Used for inserting dummy 'Lorem ipsum' text into the template
\usepackage[T1]{fontenc}
\usepackage[utf8]{inputenc}
\usepackage{amsfonts}
\usepackage{amsthm,amsmath,amssymb}

\usepackage{color}
\newcommand{\todo}[1]{\textcolor{red}{TODO: #1}\PackageWarning{TODO:}{#1!}}

% Margins
\topmargin=-0.45in
\evensidemargin=0in
\oddsidemargin=0in
\textwidth=6.5in
\textheight=9.0in
\headsep=0.25in

\linespread{1.1} % Line spacing

% Set up the header and footer
\pagestyle{fancy}
\lhead{\hmwkTitle} % Top left header
%\chead{\hmwkClass\ (\hmwkClassInstructor\ \hmwkClassTime): \hmwkTitle} % Top center head
%\rhead{\firstxmark} % Top right header
\lfoot{\lastxmark} % Bottom left footer
\cfoot{} % Bottom center footer
\rfoot{Page\ \thepage\ of\ \protect\pageref{LastPage}} % Bottom right footer
\renewcommand\headrulewidth{0.4pt} % Size of the header rule
\renewcommand\footrulewidth{0.4pt} % Size of the footer rule

\setlength\parindent{0pt} % Removes all indentation from paragraphs

%----------------------------------------------------------------------------------------
%	CODE INCLUSION CONFIGURATION
%----------------------------------------------------------------------------------------

\definecolor{MyDarkGreen}{rgb}{0.0,0.4,0.0} % This is the color used for comments
\lstloadlanguages{Perl} % Load Perl syntax for listings, for a list of other languages supported see: ftp://ftp.tex.ac.uk/tex-archive/macros/latex/contrib/listings/listings.pdf
\lstset{language=Perl, % Use Perl in this example
        frame=single, % Single frame around code
        basicstyle=\small\ttfamily, % Use small true type font
        keywordstyle=[1]\color{Blue}\bf, % Perl functions bold and blue
        keywordstyle=[2]\color{Purple}, % Perl function arguments purple
        keywordstyle=[3]\color{Blue}\underbar, % Custom functions underlined and blue
        identifierstyle=, % Nothing special about identifiers                                         
        commentstyle=\usefont{T1}{pcr}{m}{sl}\color{MyDarkGreen}\small, % Comments small dark green courier font
        stringstyle=\color{Purple}, % Strings are purple
        showstringspaces=false, % Don't put marks in string spaces
        tabsize=5, % 5 spaces per tab
        %
        % Put standard Perl functions not included in the default language here
        morekeywords={rand},
        %
        % Put Perl function parameters here
        morekeywords=[2]{on, off, interp},
        %
        % Put user defined functions here
        morekeywords=[3]{test},
       	%
        morecomment=[l][\color{Blue}]{...}, % Line continuation (...) like blue comment
        numbers=left, % Line numbers on left
        firstnumber=1, % Line numbers start with line 1
        numberstyle=\tiny\color{Blue}, % Line numbers are blue and small
        stepnumber=5 % Line numbers go in steps of 5
}

% Creates a new command to include a perl script, the first parameter is the filename of the script (without .pl), the second parameter is the caption
\newcommand{\perlscript}[2]{
\begin{itemize}
\item[]\lstinputlisting[caption=#2,label=#1]{#1.pl}
\end{itemize}
}

%----------------------------------------------------------------------------------------
%	DOCUMENT STRUCTURE COMMANDS
%	Skip this unless you know what you're doing
%----------------------------------------------------------------------------------------

% Header and footer for when a page split occurs within a problem environment
\newcommand{\enterProblemHeader}[1]{
\nobreak\extramarks{#1}{#1 continued on next page\ldots}\nobreak
\nobreak\extramarks{#1 (continued)}{#1 continued on next page\ldots}\nobreak
}

% Header and footer for when a page split occurs between problem environments
\newcommand{\exitProblemHeader}[1]{
\nobreak\extramarks{#1 (continued)}{#1 continued on next page\ldots}\nobreak
\nobreak\extramarks{#1}{}\nobreak
}

\setcounter{secnumdepth}{0} % Removes default section numbers
\newcounter{homeworkProblemCounter} % Creates a counter to keep track of the number of problems

\newcommand{\homeworkProblemName}{}
\newenvironment{homeworkProblem}[1][Problem \arabic{homeworkProblemCounter}]{ % Makes a new environment called homeworkProblem which takes 1 argument (custom name) but the default is "Problem #"
\stepcounter{homeworkProblemCounter} % Increase counter for number of problems
\renewcommand{\homeworkProblemName}{#1} % Assign \homeworkProblemName the name of the problem
\section{\homeworkProblemName} % Make a section in the document with the custom problem count
\enterProblemHeader{\homeworkProblemName} % Header and footer within the environment
}{
\exitProblemHeader{\homeworkProblemName} % Header and footer after the environment
}

\newcommand{\problemAnswer}[1]{ % Defines the problem answer command with the content as the only argument
\noindent\framebox[\columnwidth][c]{\begin{minipage}{0.98\columnwidth}#1\end{minipage}} % Makes the box around the problem answer and puts the content inside
}

\newcommand{\homeworkSectionName}{}
\newenvironment{homeworkSection}[1]{ % New environment for sections within homework problems, takes 1 argument - the name of the section
\renewcommand{\homeworkSectionName}{#1} % Assign \homeworkSectionName to the name of the section from the environment argument
\subsection{\homeworkSectionName} % Make a subsection with the custom name of the subsection
\enterProblemHeader{\homeworkProblemName\ [\homeworkSectionName]} % Header and footer within the environment
}{
\enterProblemHeader{\homeworkProblemName} % Header and footer after the environment
}

%----------------------------------------------------------------------------------------
%	NAME AND CLASS SECTION
%----------------------------------------------------------------------------------------

\newcommand{\hmwkTitle}{Image classification using Vietoris-Rips complex} % Assignment title
\newcommand{\hmwkDueDate}{Friday,\ May\ 13,\ 2016} % Due date
\newcommand{\hmwkClass}{Computational topology} % Course/class
\newcommand{\hmwkClassTime}{} % Class/lecture time
\newcommand{\hmwkClassInstructor}{Neža Mramor Kosta, Gregor Jerše} % Teacher/lecturer
\newcommand{\hmwkAuthorName}{Andrej Dolenc, Peter Us, Rok Ivanšek} % Your name

% Comands for theorems, definitions, other
\newtheorem{definition}{Definition}
\newtheorem{claim}{Claim}
%----------------------------------------------------------------------------------------
%	TITLE PAGE
%----------------------------------------------------------------------------------------

\title{
\vspace{2in}
\textmd{\textbf{\hmwkClass:\ \hmwkTitle}}\\
% \normalsize\vspace{0.1in}\small{Due\ on\ \hmwkDueDate}\\
\normalsize\vspace{0.1in}\small{\today}\\
\vspace{0.1in}\large{\textit{\hmwkClassInstructor\ \hmwkClassTime}}
\vspace{3in}
}

\author{\textbf{\hmwkAuthorName}}
\date{} % Insert date here if you want it to appear below your name

%----------------------------------------------------------------------------------------

\begin{document}

\maketitle

%----------------------------------------------------------------------------------------
%	TABLE OF CONTENTS
%----------------------------------------------------------------------------------------

%\setcounter{tocdepth}{1} % Uncomment this line if you don't want subsections listed in the ToC

\newpage
\tableofcontents
\newpage

%----------------------------------------------------------------------------------------
%	Project description
%----------------------------------------------------------------------------------------

\begin{homeworkProblem}[Project description]
As the title implies, the idea behind the project is to use the Vietoris-Rips complex for image classification. 

\todo{What is the end goal}
\end{homeworkProblem}

%----------------------------------------------------------------------------------------
%	Obtaining and preprocessing data
%----------------------------------------------------------------------------------------

% To have just one problem per page, simply put a \clearpage after each problem

\begin{homeworkProblem}[Obtaining and preprocessing data]
%Listing \ref{homework_example} shows a Perl script.

\todo{Show the images}

For our dataset we used \todo{how many classes, images}
We first resized the images down to $1296$ pixels in width and $864$ pixels in height in order to save on space and loading times. We then converted the images into grayscale color space, meaning each of the pixels only had an integer value between $0$ (black) and $255$ (white) associated with it.

We can think of images as matrices with each of the entries representing the intensity of the corresponding pixel.
\todo{Describe the process of generating matrix $X$}

\todo{Describe the preprocessing steps and the idea behind it}

\end{homeworkProblem}

%----------------------------------------------------------------------------------------
%	The classification model
%----------------------------------------------------------------------------------------

\begin{homeworkProblem}[The classification model]

\begin{definition}[Vietoris-Rips complex]
\label{defa}
Let $X$ be a a set of $m$-dimensional points $X \in \mathbb{R}^{m}$ and let $d$ be a metric. Pick a parameter $r > 0$. Construct a simplicial complex as follows:
\begin{itemize}
	\item Add a $0$-simplex for each point in $X$.
	\item For $x_{1}, x_{2} \in X$ add a $1$-simplex between $x_{1}, x_{2}$ if $d(x_{1}, x_{2}) \leq r$.
	\item For $x_{1}, x_{2}, x_{3} \in X$ add a $2$-simplex with vertices $x_{1}, x_{2}, x_{3}$ if $d(x_{1}, x_{2}), d(x_{1}, x_{3}), d(x_{2}, x_{3}) \leq r$.
	\item ...
	\item For $x_{1}, x_{2}, ... , x_{m} \in X$, add a $(m-1)$-simplex with vertices $x_{1}, x_{2}, ..., x_{m}$ if $d(x_{i}, x_{j}) \leq r$ for $0 \leq i,j \leq m$; that is, if all the points are within a distance of $r$ from each other.
\end{itemize}
The simplicial complex is called the Vietoris-Rips complex and is denoted $VR_{r}(X)$.
\end{definition}

\todo{Describe how VR cx is then used to classify images}

\todo{Explain why we only care about simplices of dimension 1}

\begin{homeworkSection}{Relation to single linkage clustering algorithm}

Our intuition tells us, that the model we build using the Vietoris-Rips complex to classify the images, produces the same results as the well known single linkage clustering algorithm. In this section we aim to prove or at least give a strong intuition that this is indeed the case.

\begin{definition}
We say that disjoint subsets $A_{1}...A_{k}$ of vertices $V$ in graph $G(V,E)$ are $k$ connected components of the graph $G$ if the following is true:
\begin{enumerate}
	\item The vertices inside $A_{i}$ are connected i.e. there exists a path between arbitrary two vertices $a, b \in A_{i}$, for every $i \in 1...k$.
	\item The sets of vertices $A_{1},...,A_{k}$ are disconnected i.e. there isn't an edges $e(a, b)$ between a pair of two points $(a,b)$ such that $a \in A_{i}, b \in A_{j}, i \neq j$.
\end{enumerate}
\end{definition}

Both algorithms take a set $X$ of $n$ samples with $m$ features as input. We can think of a sample $x$ in the set $X$ as a point in the $m$-dimensional space $x \in \mathbb{R}^{m}$. The algortihms then constructs a graph with points $X$ as the vertices ($V$) and edges $E$. The $k$ connected components in the constructed graph corespond to the classes of samples.

\begin{paragraph}{Single linkage algorithm.}
The algorithm starts with $n$ connected components (no edges in the graph). n each step the algorithm chooses the two connected components that are closest to each according to some distance metric $d$ (in our case the euclidean distance) and joins them into one by adding an edge between their closest two vertices. 

\begin{definition}
The distance $D$ between two connected components $A$ and $B$ is defined as the distance of the pair of vertices (one from $A$ and one from $B$) that are closest to each other. More formally
$$D(A, B) =  \min_{a \in A, b \in B} d(a,b).$$
\end{definition}

The algorithm stops when there are only $k$ connected components left.

\end{paragraph}

\begin{paragraph}{Vietoris-Rips classification algorithm.}
The algorithm builds a ($1$-dimensional) Vietoris-Rips complex $V_{r}(X)$ with parameter $r$. We choose the biggest $r$ such that the Vietoris-Rips complex $V_{r}(X)$ has $k$ connected components.
\end{paragraph}
\\
\\
To prove that the two algorithms indeed produce the same connected components we will first prove the next claim.

\begin{claim}\label{claima}
Let $G_{sl}(V, E_{sl})$ be a graph produced by the single linkage algorithm for finding $k$ clusters and let $d_{max}$ denote the distance between vertices in graph $G_{sl}$ that were connected in the last iteration of the algorithm. Graph $G_{sl}$ has connected components $A_{1}...A_{k}$. The graph $G_{vr}(V, E_{vr})$ induced by the Vietoris-Rips complex $VR_{d_{max}}(V)$ has the same connected components $A_{1}...A_{k}$.
\end{claim}

\begin{proof}
We prove the Claim~\ref{claima} by induction on the steps in the single linkage algorithm. We start with a set of vertices $V$. Let $j$ denote the step (iteration) of the algorithm, $e_{j}$ the edge added in $j$-th step and $d_{j}$ its length. We claim that at each $j$ the graph $G_{sl}^{j}$ constructed by the algorithm up to that point, has the exact same conected components as $G_{vr}^{j}$, that is the graph induced by the Vietoris-Rips complex $VR_{d_{j}}(V)$.

\begin{paragraph}{Base case.}For $j=0$ this is obvious, since this is the initial state of the algorithm. Both graphs $G_{sl}^{0}$ and $G_{vr}^{0}$ consist only of vertices $V$. For $j=1$ the algorithm adds the smallest edge $e_{1}$ out of all possible candidates and builds a graph $G_{sl}^{1}$. Edge $e_{1}$ has length $d_{1}$. It is obvious that $VR_{d_{1}}(V)$ will induce a graph $G_{vr}^{1}$ that will also only contain edge $e_{1}$, since no other pairwise distance between vertices $V$ is smaller.
\end{paragraph}

\begin{paragraph}{Induction step.}
Here we show that if for some $j$ our claim holds, it will also hold after another iteration of the algorithm i.e. for $j+1$. In $(j+1)$-th iteration, the algortihm finds the edge $e_{j+1}$ with length $d_{j+1}$ and adds it to the graph. By the definition of the algortihm $e_{j+1}$ is the smallest such edge that connects (joins) two seperate connected components. This means that every other edge $e'$ with length $d' < d_{j+1}$ would not join connected components, but would instead just connect some vertices, that are both allready in the same connected component. From the definition of the Vietoris-Rips complex we can see that in the graph $G_{vr}^{j+1}$ there will only be one new edge that will join two seperate connected components, and that will be exactly edge $e_{j+1}$. All the other extra edges that will be added in $G_{vr}^{j+1}$, but do not appear in  $G_{sl}^{j+1}$ will only connect vertices in the allready existing connected components of the graph $G_{vr}^{j}$. Since by our induction hypothesis graphs $G_{sl}^{j}$ and $G_{vr}^{j}$ had the same connected components and we joined two of the same connected components in both graphs this means that the graphs $G_{sl}^{j+1}$ and $G_{vr}^{j+1}$ also have the same connected components.
\end{paragraph}\\
\\
We have proven that the graph $G_{sl}^{j}$ constructed in $j$-th iteration of the single linkage algorithm indeed contains the same connceted commponents as the graph $G_{vr}^{j}$ induced by $VR_{d_{j}}(V)$ for an arbitrary $j$. This also prooves Claim~\ref{claima}.
\end{proof}

Using Claim~\ref{claima} we see that the connected components in $G_{vr}$ and $G_{sl}$ are indeed the same. We need to take into account that the Vietrois-Rips algorithm takes the biggest such $r$, so that the graph has $k$ connected components, so $r > d_{max}$. But we can quickly see that the extra edges in the graph induced by $V_{r}(V_{sl})$ will not change the connected components. After all we allready have $k$ connected components in $G_{vr}$. To join any two together would mean a violation of a fundemental rule of the algorithm.

\end{homeworkSection}

\begin{homeworkSection}{Computational complexity}

  Let us now consider the computational complexity of our model. Since we are only interested in Vietoris-Rips complexes $VR_r$ with simplices of dimension $1$, the simplest approach to construct such complex requires us to check the distance between every pair of vertices $(x_1, x_2) \in X \times X$, adding such pair to the final complex if the distance $d(x_1, x_2) \le r$. In worst case the algorithm would have to return all distinct pairs of vertices, meaning construction of $VR_r(X)$ requires $O(n^2)$ time and consumes $O(n^2)$ space, where $n$ is the number of vertices in $X$.

  The problem is that we don't know the appropriate value for the parameter $r$. Recall that we are interested in finding biggest $r$, such that the Vietoris-Rips complex $VR_{r}$ has as many connected components as there are distinct classes of images. Let $r_{max}$ denote the largest distance between two vertices from $X$. Note that $r$ we are looking for will always be bounded on the interval $[0, r_{max}]$, and it will furthermore be exactly one of the distances between some pair of vertices. Thus we only have $n^2$ different possible values of $r$ to check, and if we sort them by size and use binary search to find the right one, we can do it in $O(n \log n)$ time and $O(n^2)$ space. To count the number of connected components obtained with each of different $VR$ complexes, we can use a \texttt{union-find} algorithm, which roughly adds a $O(n^2)$ time to each run.

  With this the final time complexity of our approach is $O((n^2 + n^2)\log n^2) = O(n^2 \log n)$ using $O(n^2)$ space. \todo{Double check the time complexity of union-find}

  Contrast this with the computational complexity of single-linkage clustering, which with a clever implementation can in optimal case produce solution in $O(n^2)$ time and $O(n)$ space. \todo{Find reference from wikipedia}.

\end{homeworkSection}


%\begin{center}
%\includegraphics[width=0.75\columnwidth]{example_figure} % Example image
%\end{center}


\end{homeworkProblem}

\begin{homeworkProblem}[Results]
\todo{Explain we get same results as with S-L clustering, and with that the same problems as S-L clustering.}

\todo{Present MDS graph of our dataset after applying preprocessing, and emphasize different classes/recognized clusters.}

\todo{Explain that simplices of larger dimension are just about useless}

\todo{Explain what datasets our solution actually works with (e.g. determining position of objects on image)}

\todo{Extra tests: images from midpoints of edges, barycenters of simplices, \ldots}

\end{homeworkProblem}

\begin{homeworkProblem}[Summary]

\todo{Brief summary}

\todo{Further work (if there even is any), what we didn't explore.}
\end{homeworkProblem}

%----------------------------------------------------------------------------------------
%	Testing
%----------------------------------------------------------------------------------------

%----------------------------------------------------------------------------------------

\begin{thebibliography}{9}

\bibitem{lamport94}
  Leslie Lamport,
  \emph{\LaTeX: a document preparation system},
  Addison Wesley, Massachusetts,
  2nd edition,
  1994.

\end{thebibliography}

\end{document}
